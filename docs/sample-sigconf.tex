
\documentclass[sigconf]{acmart}
\settopmatter{printacmref=false}


\AtBeginDocument{%
  \providecommand\BibTeX{{%
    \normalfont B\kern-0.5em{\scshape i\kern-0.25em b}\kern-0.8em\TeX}}}




\begin{document}

\title{Name in progress... }

\author{Michael Frank}
\email{michael.frank@stud.hs-flensburg.de}
\affiliation{%
  \institution{Hochschule Flensburg}
  \streetaddress{Kanzleistraße 91–93}
  \city{Flensburg}
  \state{Schleswig-Holstein}
  \country{Deutschland}
  \postcode{24943}
}

\author{Nico Hohm}
\email{nico.hohm@stud.hs-flensburg.de}
\affiliation{%
  \institution{Hochschule Flensburg}
  \streetaddress{Kanzleistraße 91–93}
  \city{Flensburg}
  \state{Schleswig-Holstein}
  \country{Deutschland}
  \postcode{24943}
}

%%
%% The abstract is a short summary of the work to be presented in the
%% article.
\begin{abstract}
  Abstract in progress ...
\end{abstract}

%%
%% The code below is generated by the tool at http://dl.acm.org/ccs.cfm.
%% Please copy and paste the code instead of the example below.
%%
\begin{CCSXML}
<ccs2012>
 <concept>
  <concept_id>10010520.10010553.10010562</concept_id>
  <concept_desc>Computer systems organization~Embedded systems</concept_desc>
  <concept_significance>500</concept_significance>
 </concept>
 <concept>
  <concept_id>10010520.10010575.10010755</concept_id>
  <concept_desc>Computer systems organization~Redundancy</concept_desc>
  <concept_significance>300</concept_significance>
 </concept>
 <concept>
  <concept_id>10010520.10010553.10010554</concept_id>
  <concept_desc>Computer systems organization~Robotics</concept_desc>
  <concept_significance>100</concept_significance>
 </concept>
 <concept>
  <concept_id>10003033.10003083.10003095</concept_id>
  <concept_desc>Networks~Network reliability</concept_desc>
  <concept_significance>100</concept_significance>
 </concept>
</ccs2012>
\end{CCSXML}

%%
%% Keywords. The author(s) should pick words that accurately describe
%% the work being presented. Separate the keywords with commas.
\keywords{Open source development, Open source, Software development, Development initiative,
		Browser extension, Chrome extension, GitHub, Blockchain, Ethereum}

%%
%% This command processes the author and affiliation and title
%% information and builds the first part of the formatted document.
\maketitle
\pagestyle{plain}


\section{Introduction}
\subsection{Open source development on GitHub}
Open source development is a type of software development in which a decentralised and collaborative 
community develops software publicly and transparently. Open source development enables the creation 
of innovative and free software through the collaboration of many people, and it also provides free access 
to the software for everyone \cite{shaikh2017governing, redhat2021ops}. Some examples for such significant 
projects are the Linux Kernel \cite{linux2021ops} and the Mozilla Firefox browser \cite{mozilla2021ops}, which are 
used by many people  every day. A particular difficulty in open source development is the coordination of the many 
developers who contribute to the development of the projects with their own extensions or improvements. Changes have 
to be tracked, traced and, if necessary, reversed \cite{shaikh2017governing}. These problems occur not only in open 
source development, but also in normal software development. As in normal software development, version control is 
used to solve these problems \cite{shaikh2017governing, ulrich2020dev}. \\ \\
In particular, the version control protocol Git 
is ideally for open source development, because with Git it is possible to have several distributed remotes that can 
access and manage the same source code \cite{git2021scm, ulrich2020dev}. Projects that are coordinated via git are 
called repositories. The developers have a local copy of the repository on their systems and can push their changes to or 
pull the current status from the main repository. These actions are coordinated via a so-called git server, which has to be 
hosted somewhere so that the developers can work with it \cite{git2021scm}. It is important that this server is permanently 
accessible, otherwise the actions mentioned will no longer work. and coordination of the repository will be 
interrupted \cite{ulrich2020dev}. Hosting services such as GitHub were created so that such problems can be prevented 
and not everyone has to set up their own git server if they want to start an open source project \cite{ulrich2020dev, git2021hub}.
GitHub is the largest git hosting provider today, with over 56 million registered developers and over 100 million repositories.
\cite{git2021hub} The service is also used by large IT corporations such as Microsoft, Facebook and Google and hosts a large 
number of the largest and most important open source repositories \cite{git2021stars}.\\ \\
Normally, an open source project on GitHub starts with a user creating a repository for it. The user who created the 
repository is the owner of it. This user has full control over the repository and can push changes, decide which changes are 
accepted (merged) and even delete the repository. In addition, he can add so-called collaborators to the repository, who have 
read and write rights in the repository \cite{git2021rights}. Normal users who do not have collaboration rights can contribute to 
the open source project by creating a pull request with their change. Another user with the necessary rights can then decide 
whether the change is useful or not. Depending on his decision, he can merge (accept) or reject (reject) the changes. 
The described process is the typical approach to how the community develops for an open source project.


\subsection{Problem}
The described workflow for managing repositories on GitHub but also on other hosting services such as GitLab or Bitbucket has some significant disadvantages. Firstly this approach is not necessarily decentralised or democratic. Very few people usually have the necessary rights to merge pull requests, and they can decide over the head of the general community whether to merge or reject a pull request. So it doesn't matter what the general community thinks, as long as the administrators have a different opinion. Rejecting good or useful pull requests is bad, but not a direct threat to the project. The opposite is to merge a critical bug into the main repository, which can cause enormous damage, as in the example of the Heartbleed bug in the Open SSL repository \cite{ioriheartbleed}. This danger exists mainly because it only takes one person with the necessary rights to overlook the bug and decide to merge the flawed pull request. A less critical problem, but one that affects many of the open source repositories that are not supported by companies or large communities, is the lack of funding and thus further development. Many projects live on developers who work on these projects in their spare time, which means that further development and maintenance of this repositories can sometimes stagnate or even cease completely.


\subsection{Solution}

\section{Related work}
Related work falls into three areas:
- Drei Bereiche, Governing, Voting, Initiative
- Torbens (Dezentralisierter Ansatz)
- GOVERNING OPEN SOURCE SOFTWARE THROUGH COORDINATION PROCESSES 
- Bounties in Open Source Development on GitHub: A Case Study of Bountysource Bounties

\section{Protocol}
- Wieso gibt es zwei Protokolle?
\subsection{First iteration}
- Jeder konnte ein Voting starten
- Voting ohne Kommentar
- Bild des Protocolls

\subsection{Second iteration}
- Admin starten Bounty
- Restliches Protocl

\section{Implementation}


\section{Discussion}
- Was sehe ich für Probleme

\section{Conclusion and future work}
- Anbindung an Torbens 




%%
%% The next two lines define the bibliography style to be used, and
%% the bibliography file.
\bibliographystyle{ACM-Reference-Format}
\bibliography{sample-base}

%%
%% If your work has an appendix, this is the place to put it.
\appendix

\end{document}
\endinput
%%
%% End of file `sample-sigconf.tex'.
